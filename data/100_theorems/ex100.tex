\documentclass{article}

\usepackage[margin=1.0in,footskip=0.25in]{geometry}

%\usepackage[english]{babel}
\usepackage{naproche}
\usepackage{amssymb}

\begin{document}
\setlength\parindent{0pt}

\newcommand{\Prod}[3]{#1_{#2} \cdots #1_{#3}}
\newcommand{\Seq}[2]{\{#1,\dots,#2\}}
\newcommand{\FinSet}[3]{\{#1_{#2},\dots,#1_{#3}\}}
\newcommand{\Primes}{\mathbb{P}}
\newcommand{\pow}{{\cal P}}
\newcommand{\range}{\operatorname{ran}}
\newcommand{\inv}[1]{#1^{-1}}
\newcommand{\sset}[2]{\{#1\}_{#2}}
\newcommand{\sumgeom}[2]{\sum_{0 \leq i < #2} {#1}^i}
\newcommand{\sumarith}[3]{\sum_{i = 1}^{#3}(#1 + #2 \cdot i)}

$ q^{2} = p $ for no positive rational number $ q $.

$ q^{2} = p $ pour aucun nombre rationnel positif $ q $.

$ q^{2} = p $ für keine positive rationale Zahl $ q $.

$ q^{2} = p $ para nenhum número racional positivo $ q $.


The collection of prime natural numbers is infinite.

La collection d'entiers naturels primaires est infinie.

Die Sammlung von unteilbaren natürlichen Zahlen ist unendlich.

O collection de números naturais primos é infinito.


Let x, y be sets. $ x $ and $ y $ are equinumerous iff there exists a injective map from $ x $ to $ y $ and there exists an injective map from $ y $ to $ x $.

Soient x, y des ensembles. $ x $ et $ y $ sont équinombreux si et seulement si il existe une correspondance injective de $ x $ à $ y $ et il existe une application injective de $ y $ à $ x $.

Seien x, y Mengen. $ x $ und $ y $ sind gleichzahlig wenn und genau dann wenn es eine injektive Abbildung aus $ x $ nach $ y $ gibt und es eine injektive Abbildung aus $ y $ nach $ x $ gibt.

Deixe x, y ser conjuntos. $ x $ e $ y $ são equinumeiros se e só se existe uma função injetiva de $ x $ a $ y $ e existe uma aplicação injetiva de $ y $ a $ x $.


For all finite sets $ X $ and all natural numbers $ n $, if $ |X| = n $, then $ \pow(X) $ is finite and $ |\pow(X)| = 2^{n} $.

Pour tous les ensembles finis $ X $ et tous les entiers naturels $ n $, si $ |X| = n $, alors $ \pow(X) $ est fini et $ |\pow(X)| = 2^{n} $.

Für alle endlichen Mengen $ X $ und alle natürlichen Zahlen $ n $, wenn $ |X| = n $, dann ist $ \pow(X) $ endlich und $ |\pow(X)| = 2^{n} $.

Para todos os conjuntos finitos $ X $ e todos os números naturais $ n $, se $ |X| = n $, então $ \pow(X) $ é finito e $ |\pow(X)| = 2^{n} $.


Let $ s, t $ be real numbers such that $ s < t $. Then there exists a real number $ z $ such that $ s < r < t $.

Soient $ s, t $ des nombres tel que $ s < t $. alors il existe un nombre $ z $ tel que $ s < r < t $.

Seien $ s, t $ reelle Zahlen derart dass $ s < t $. dann gibt es eine reelle Zahl $ z $ derart dass $ s < r < t $.

Deixe $ s, t $ ser números tal que $ s < t $. então existe um número $ z $ tal que $ s < r < t $.


Let $ M $ be a set. Then there exists no surjection from $ M $ onto the powerset of $ M $.

Soit $ M $ un ensemble. alors il n'existe aucune surjection de $ M $ sur l'ensemble puissance de $ M $.

Sei $ M $ eine Menge. dann gibt es keine Surjektion aus $ M $ auf die Potenzmenge $ M $.

Deixe $ M $ ser um conjunto. então não existe nenhuma sobrejecção de $ M $ sobre o conjunto de potência de $ M $.


$ \sumgeom{x}{n} = \frac{1 - x^{n}}{1 - x} $ for all natural numbers $ n $.

$ \sumgeom{x}{n} = \frac{1 - x^{n}}{1 - x} $ pour tous les entiers naturels $ n $.

$ \sumgeom{x}{n} = \frac{1 - x^{n}}{1 - x} $ für alle natürlichen Zahlen $ n $.

$ \sumgeom{x}{n} = \frac{1 - x^{n}}{1 - x} $ para todos os números naturais $ n $.


$ \sumarith{a}{d}{n} = n \cdot ( a + \frac{(n + 1) \cdot d}{2}). $.

$ \sumarith{a}{d}{n} = n \cdot ( a + \frac{(n + 1) \cdot d}{2}). $.

$ \sumarith{a}{d}{n} = n \cdot ( a + \frac{(n + 1) \cdot d}{2}). $.

$ \sumarith{a}{d}{n} = n \cdot ( a + \frac{(n + 1) \cdot d}{2}). $.


Let $ m, n $ be natural numbers such that $ m < n $. Then the greatest common divisor of $ m $ and $ n $ is the greatest common divisor of $ n-m $ and $ m $.

Soient $ m, n $ des entiers naturels tel que $ m < n $. alors le plus grand commun diviseur de $ m $ et de $ n $ est le plus grand commun diviseur de $ n-m $ et de $ m $.

Seien $ m, n $ natürliche Zahlen derart dass $ m < n $. dann ist der größte gemeinsame Teiler $ m $ und $ n $ der größte gemeinsame Teiler $ n-m $ und $ m $.

Deixe $ m, n $ ser números naturais tal que $ m < n $. então o máximo divisor comum de $ m $ e $ n $ é o máximo divisor comum de $ n-m $ e $ m $.


Assume $ A \subseteq \mathbb{N} $ and $ 0 \in A $ and for all $ n \in A $, $ n + 1 \in A $. Then $ A = \mathbb{N} $.

Supposons que $ A \subseteq \mathbb{N} $ et $ 0 \in A $ et pour tout $ n \in A $, $ n + 1 \in A $. alors $ A = \mathbb{N} $.

Wir nehmen an, dass $ A \subseteq \mathbb{N} $ und $ 0 \in A $ und für alle $ n \in A $, $ n + 1 \in A $. dann $ A = \mathbb{N} $.

Admitemos que $ A \subseteq \mathbb{N} $ e $ 0 \in A $ e para todo $ n \in A $, $ n + 1 \in A $. então $ A = \mathbb{N} $.


\end{document}

